\section{Background Information \& Research}

\subsection{What is Fuzzy Logic?}
Fuzzy logic is a ``natural'' way of expressing uncertain or qualitative information \cite{albertos1998fuzzy}. It is a form of logic that deals with approximate reasoning, as opposed to fixed, exact values, like those found in classical logic (where we may only have properties being true, or false). Instead of these strict truth values, fuzzy logic systems have a range of truth, between 0 and 1. This makes fuzzy logic much better for handling and sorting data, and is an excellent choice for many control system applications, due to the way it mimics human control logic. Lotfi Zadeh, who formalised fuzzy logic in 1965, states that the key advantages of fuzzy logic are that it allows us to make rational decisions in environments of imprecision, uncertainty, and partiality of truth, and to perform a wide variety of physical and mental tasks, without any measurements or computations \cite{zadeh1999computing}.\\[2mm]
\noindent 
In a classical set, the membership, $\mu_A(x)$ of $x$, of a set, $A$, in universe, $X$, is defined:

\begin{center}
\vspace{-3mm}
$ 
\mu_A(x) = \left\lbrace
\begin{array}{ll}
1, & $iff x $\in$ A$      \\
0, & $iff x $\notin$ A$   \\
\end{array} \right\rbrace $ 
\end{center}
\vspace{-2mm}
\noindent 
That is, the element is either in the set, or not. In a fuzzy set, however, we have grades of membership, which are real numbers in the interval, $\mu_A(x) \in [0,1]$. Every member of a set has a membership grade to that set, depicting how true the property represented by that set is, for the given member \cite{zadeh1965fuzzy}. The traditional syntax for representing members of a fuzzy set is given below (although a full working knowledge of fuzzy logic theory is not necessary for this project).
\vspace{-2mm}
\begin{center}
$A = \mu_A(x_1)/x_1 + ... + \mu_A(x_n)/x_n$
\end{center}
\vspace{-2mm}
The easiest way to observe the merits of fuzzy logic are to look at terms that we humans use in our everyday life, and attempt to map these are crisp functions. For instances, terms like ``hot'', ``cold'', ``tall'', ``short'', are all terms that we understand very well, and use often. However, if we were asked to give \textit{exact} values for tallness, or shortness, we would not be able to. At what cut-off point would a person change from being considered short, to being considered tall? Fuzzy logic helps to alleviate these impossible choices, by having varying differing degrees of membership, to certain properties. The example in figure \ref{fig:fuzExample} shows this using three linguistic variables to describe the height of a person. Instead of at one point being either tall, short, or medium height, we, at all times, belong to all properties, to a differing degree. 

\begin{figure}[ht!]
\begin{center}
\includegraphics[width=0.6\textwidth]{images/fuzExample.png}
\end{center}
\vspace{-5mm}
\caption{A fuzzy set depicting ``height''}
\label{fig:fuzExample}
\end{figure}
\noindent
For instance, at the point labelled $z$, in the sets in figure \ref{fig:fuzExample}, we belong in the ``Short'' set, to degree 0.7, we belong in the ``Medium'' set, to degree 0.3, and we belong in the ``Tall'' set, to degree 0.0. This is, naturally, much more precise than simply saying we are ``Short'', ``Medium'', or ``Tall''.

\subsection{Existing Systems}
\label{sec:existing-systems}

Fuzzy logic has been around for almost 50 years now, and, with the rising age of the computer, it would be alarming if no software systems for its usage were in circulation. Luckily, this is not the case, and there are many examples of software systems focusing on the use of fuzzy logic, of which many different approaches have been attempted, to varying degrees of success. In this section, a number of these software systems will be evaluated, to discern their positive and negative qualities, to help improve the design of the project presented in this report. 
\paragraph{MATLAB Fuzzy Toolbox}\ \\The first system to be explored is MATLAB's fuzzy toolbox, an add-on for the MATLAB software suite, to work with fuzzy sets and systems. This toolbox provides everything required to create type-1 fuzzy sets and systems, with relative ease. The main advantage it has over most other systems is that it has a graphical user interface, which makes a tasks like working with fuzzy sets (that require a lot of visualisation and updating in real time) much simpler. There is also an extensive library of documentation and tutorials available for both MATLAB, and this specific toolbox, that help novices to get acquainted with the system. These things both help to make the system very easy to use, and novice friendly.\ \\
\ \\
Unfortunately, these positives do not outweigh the major disadvantage of MATLAB, and the fuzzy toolbox; which is that are pieces of proprietary software. This means that a novice to the field of fuzzy logic would have to invest a considerable sum of money, before they could even begin using the software. Whilst the system does have extensive documentation, and the user would be able to understand and use the system with relative ease, a piece of software does not require a large price tag to achieve this level of functionality and support. Another disadvantage of the MATLAB fuzzy toolbox is that is it not a dedicated piece of software, and is instead a limited subsection of the greater software of MATLAB. This means that the potential for extensibility is much less likely, as updates to the encompassing software would be deemed more important. It could even be argued that the installation of the MATLAB software, and then the installation of further software could be confusing to some novice users, which further alienates them.

\paragraph{FuzzyToolkitUoN}\ \\
\paragraph{XFuzzy}\ \\
\paragraph{fuzzyTECH}\ \\



{\color{red}
Talk about fuzzy toolkituon, matlab, xfuzzy (brief), fuzzytech(brief), mention their positions and negatives in terms of the key flaws, and novice users.\ \\
\ \\
Then talk about my project and how this will counter the specific issues raised.
}


\subsection{Platforms and Tools}
{\color{red}\begin{enumerate}
\item Languages used (R, Javascript)
\item Web technologies (tools, languages)
\item Shiny r to html
\item Bootstrap
\item jquery
\item Good user interfaces
\end{enumerate}}

