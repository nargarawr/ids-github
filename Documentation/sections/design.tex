\section{System Designs}
In this section, all of the design aspects of this system have been detailed and justified. 

\subsection{UI Design}
As the purpose of this project was to aid in the accessibility and usability of fuzzy logic, to users of all skill levels, the user design of the user interface was extremely important. The difference between a good piece of software, and a great piece of software, can easily be the interface they provide. It is exactly i's interface that makes FuzzyToolkitUoN an inadequate software system for specifying fuzzy sets, and the exact thing this project aims to overcome. There were two main iterations to the user interface design, the first was a simple attempt to include all information on the page, in an easily viewable format. The second iteration was a refinement of the first stage, in which design principles were applied, and feedback was gathered from potential end users. \ \\
\ \\
It is worth noting that having all the different tasks of constructing a fuzzy system (specifying inputs, specifying outputs, specifying rules, evaluating the system, and file input and output) are all distinct tasks, and there is no reason for them to be together at any point. This is why, regardless of design, these tasks are all separated and are on different pages, or tabs, of the website. Further to this, the designs presented below cover only the input variable creation page and the rule creation page, as the input creation page is an exact replica of the output variable creation page, and the remaining pages or tabs of the website do not require much level of design, as they are relatively small.

\subsubsection{First Iteration}
%	{\color{red} tabbed based system with long horizontal variables. maintaining large systems becomes more difficult. something about hci software science }

\begin{figure}[ht!]
\begin{center}
\includegraphics[width=0.8\textwidth]{images/firstItInputs}
\end{center}
\caption{The first iteration of the design of the input creation page}
\label{fig:design-firstIterationInputs}
\end{figure}
\noindent 
The first iteration of the input creation page listed the navigation to the left hand side of the page, as this is a standard positioning for navigation, and the users would, most likely, be accustomed to this. \ \\
\ \\
The separation of different tasks to different sections of the website is very important for the construction of a fuzzy system, as there are many individual elements, and a lack of separation of these would cause a great deal of confusion for the user, as there would be a huge number of elements on the screen at one time.\ \\
\ \\
The design displays the core information of the variable on the far left (including name and range), as this is where the user's eye would fall first. The next set of information (the functions contained within the variable), as displayed further to the right, as the user would look to here after reading the initial information. The users have the option to edit or delete any function that they have created, granting them complete freedom over the system and everything they do within it. They also have the option to add a new membership function (which is a green button, to represent creation, which brings up a modal window to be used to construct the membership function), or delete the current variable (which is a red button, to represent destruction). \ \\
\ \\
This design uses long horizontal representations of the variables, so a large number of them can be on the screen at the same time, as they take up little vertical space. This means the users can view many of their variables at the same time, and make edits as necessary. Colour is used sparingly throughout the page, so that it can be used effectively for highlighting important information or elements, so they are easily visible to the user.

\begin{figure}[ht!]	
\begin{center}
\includegraphics[width=0.8\textwidth]{images/firstItRules}
\end{center}
\vspace{-5mm}	
\caption{The first iteration of the design of the rule creation page}
\label{fig:design-firstIterationRules}
\end{figure}
\noindent 
The rule creation page also follows a horizontal theme, which is especially effective for the rules, as there could potentially be a long list of them. The creation of the rules, and their displaying, are two clearly distinct segments to the page, which means there is no confusion between the two, and the process remains a simple one.\ \\
\ \\
In keeping with the standards of the previous page, the ``Add'' button is coloured green, to represent creation, and the ``Delete'' buttons present by each rule are coloured red, to represent destruction. These clear colour separations are useful to the user, as they do not even need to read a button to gain an understand of what it will do, and they can be more careful to not make mistakes.\ \\
\ \\
As can be seen, the navigation and header of the page have not changed between pages, which promotes a consistent style amongst the page, and helps to remind the user they are still within the same system, even if they are completing a different task. 

\subsubsection{Second Iteration}

\begin{figure}[ht!]
\begin{center}
\includegraphics[width=0.8\textwidth]{images/secondItInputs}
\end{center}
\caption{The second iteration of the design of the input creation page}
\label{fig:design-secondIterationInputs}
\end{figure}
\noindent

\paragraph{General Improvements Made}\ \\
The second iteration of the input creation page design includes many improvements over the first iteration. Specifically, this second iteration dealt with the issues the first iteration presented, applied well known and well documented usability heuristics and design principles. \ \\
\ \\
The largest change to the design was the re-working of the navigation to be entirely tabular, and to replace the old navigation area with system wide parameters. This tabbing helps to split the different tasks of the system, as most users are familiar with the concept of tabs, due to their popularity within internet browsers. The system wide parameters are now displayed in place of the old navigation, which means they can be changed, regardless of the tab the user is on. This is especially convenient when the user comes to evaluate the system, as it makes tuning much quicker, and all permutations of parameters can be evaluated with ease.\ \\
\ \\
The overall structure and displaying of the variables has also been drastically changed in this second iteration of design. The reason for this was that, whilst the old design promoted a large number of variables on the screen at any one time, this would severely increase the cognitive load on the user, and they could quickly become confused.\ \\
\ \\
The new design works by having expandable and collapsible variables. When in the collapsed view, the variable takes up a very small space (allowing for the display of many of them at once), and in the expanded view, all of the information of the variable is present (like in the previous design). The ideal number of elements on a page is 7$\pm$2 \cite{miller1956magical}, which can easily be exceeded with these new collapsed variables. However, the reason this design works, is because when the variable is collapsed, the user feels like it is completed, and they no longer need to concern themselves with it. So whilst this new design may look more cluttered, it significantly helps to reduce the cognitive load of the user, as they can create a variable, and then essentially forget about it.\ \\
\ \\
To fit with the consistency of the other pages, the buttons for the editing and deletion of membership functions have also been changed. The edit button is now blue, which is the system wide colour for editing, and the delete buttons are now red, the system wire colour for deletion. This keeps the website consistent, and makes the functionality of these buttons easily identifiable.\ \\
\ \\
A graphical representation of each variable is now also displayed in the expanded view, allowing the user to quickly interpret how their system is progress, and whether it looks as they expected. This is a very powerful tool, as users will be able to spot errors much sooner than if this were not present.\ \\
\ \\
Some other smaller changed include the moving of the ``Add'' button from the bottom right of the page, to the top left. The reason for this is that, in the first iteration of design, the add button would constantly move down the page as variables are added, and a moving button can be frustrating and confusing for some users. Another change is the addition of a short name for the software, with a longer tag-line underneath this, on the top of the page. This shortened title allows for users to refer to system with ease, which means they can search for help for the system if necessary, and simply discuss the system with other potential users. The longer tag-line allows for a brief description of the system, so the user knows whether it will be able to accomplish the tasks they have set out to achieve.


\begin{figure}[ht!]
\begin{center}
\includegraphics[width=0.8\textwidth]{images/secondItRules}
\end{center}
\caption{The second iteration of the design of the rule creation page}
\label{fig:design-secondIterationRules}
\end{figure}

The rules creation page also received heavy changes in the second iteration. The first changes made were the addition of some functionality that was not present on the initial design, but was necessary. For instance, the ability to edit rules once they had been created, the ability to create negated terms (If food is \textit{not} good) and the addition of the help button.\ \\
\ \\
However, the biggest change is the removal of the ability to create rules, on this initial page. Instead, as with the creation of membership functions, this has been moved into a separate window that is launched when the user clicked the ``Add'' button. This reduces the clutter on the page, and means the rule creation functionality is only accessed when the user requires it.\ \\
\ \\
In addition, an extra feature was included in the form of a rule table that could be displayed if the user was using a system with two inputs, and one output (a relatively common set up for a fuzzy system). If this were the case, a table would be displayed, mapping the inputs to the outputs, in an intuitive graphical manner, as shown in figure \ref{fig:ruleTable}.

\begin{figure}[ht!]
\begin{center}
\includegraphics[width=0.6\textwidth]{images/ruletable}
\end{center}
\caption{An example of a rule table that would be displayed if the user had two inputs (heart rate and temperature) and a single output (urgency)}
\label{fig:ruleTable}
\end{figure}

\paragraph{Heuristic Evaluation}\ \\
The new design also went through a heuristic evaluation \cite{nielsen1990heuristic}, using the 10 Usability Heuristics laid out by Jakob Nielson, and the Golden Rules for Design, laid out by Ben Shneiderman \cite{shneiderman2005designing}. Both of whom are extremely well known and very influential in the field of Human Computer Interaction, and their heuristics define some of the most basic, but most important properties a user interface should possess. 

\begin{enumerate}
\item Visibility of system status/Offer informative feedback\\
It is important that the users of a system are always updated as to what is happening within the system. This will be implemented through the use of JavaScript alert messages to alert the user to any errors they have made, or any important changes they have made.
\item User control and freedom\\
The user should be in full control of their experience of the system at all times, and thus a fluid and flexible navigation system is important. This is the purpose of the tabular layout of the website, as the user is free to travel to which pages they wish, in whichever order. More detail on this can be found in section \ref{subsec:nav}.
\item Consistency and standards/Strive for consistency\\
A good interface should also be consistent in it's design, and follow platform standards. As you can see from the design of the input creator and the rule creator pages, the consistent tabular layout is consistent throughout the website, and colours such as blue, green, and red have clearly defined purposes, which stay consistent throughout the website.
\item Error prevention/ Help users recognize, diagnose, and recover from errors\\
Unfortunately, with a system that is entirely dictated by user input, it is very difficult to prevent the user from making errors. However, measures will be put in place to ensure these errors do not affect the system. For instance, there is no way to enforce the user of a system to enter a number, but an informative error message will be displayed, telling the user this is not valid, and what should be done to rectify the issue.
\item Recognition rather than recall/Reduce short-term memory load\\
As mentioned multiple times already, the system is designed to reduce cognitive load on the user as much as possible. This is done by reducing the number of elements on the page at any one time, and by using modal windows for the creation of new membership functions and rules, so their creation and their display are distinct.
\item Aesthetic and minimalist design\\
To reduce any possible distractions for the user, the system is designed in a minimalistic fashion, using colour very sparingly, and sticking to neutral shades for background elements. The only colour used in the system is on the graphs drawn, and on the important buttons the user will be pressing. These buttons are essentially colour coded so the user is aware of their functionality, without even having to read them. Further to this, the use of modal windows to essentially hide functionality greatly helps to reduce clutter on the pages of the website, giving it a much cleaner look.
\item Help and documentation\\
Due to the goal of being as easy as use as possible, the system is to be designed with a dedicated help system, built in. The advantage of this is that the user is able to access help without having to leave the application itself. Help is accessed via the large green help button present on every page, which is easily visible, and provides concise and helpful information for the user.
\item Design dialogue to yield closure\\

\item Offer simple error handling\\

\item Permit easy reversal of actions\\

\item Support internal locus of control\\


\end{enumerate}

As the final stage of the heuristic evaluation process, the 13 sins of New Media Design \cite{golombisky2013white} were also observed, to ensure an optimal design. The second iteration of design has taken these sins into account and has avoided those that were appropriate. Specifically, the website features no bulky borders, a generous use of margins, left alignment of elements (as opposed to centring), and an avoidance of a busy background, opting instead for a very simply white background. All of the aforementioned design choices could have easily cause distractions for the user, and damage their enjoyment of the system.











% Jakob Nielsson usability heuristics 
 

% golden rules for design \cite{shneiderman2005designing}
%	Strive for consistency, Offer informative feedback, Design dialog to yield closure, Offer simple error handling, 
%	
%	

%software science
%magical number seven (\cite{miller1956magical}) "


\subsection{Navigation/Control Flow Design}	
\label{subsec:nav}	
%{\color{red} Section to detail the navigation and flow of the system, and all design choices as a result of that, compare first and second iterations }
% back and forth bitch, back and forth

\subsection{Internal Design}
The user interface of a system is a huge part of the design process, and is extremely important. However, this is not the only part of the system that requires designing; the internal workings of the system also require a lot of thought. 
%{\color{red} Show how the backend and foreground interact, and how this was designed (see picture from presentation) }
