\section{Motivation}
The motivation of this project is a simple one: to produce a fuzzy logic software system that is easy to access, and easy to use, to help promote the wider adoption of fuzzy logic. The problem with systems currently available is that they suffer from one of the two following pitfalls: difficulty of use or difficulty of access. This means that novices can find it very difficult to get into the field, the software available does not facilitate productive use, and even experts can be held back by the software they are using. Some specific issues include: locating systems to use, complex installation processes, cost to the user, unintuitive user interface, or a requirement of (a considerable amount of) prior knowledge. 
\ \\
\ \\
As part of my second year group project at the University of Nottingham, I worked on an R Package called FuzzyToolkitUoN. The goal of this system was to expand upon work completed by the Intelligent Modelling and Analysis group\footnote{\url{http://ima.ac.uk/}}, to facilitate the use of fuzzy logic within the R programming language \cite{wagner2011fuzzy}. Whilst working on this project, a large amount of research into existing fuzzy logic software systems was conducted, and it was during this research period that the two key commons flaws were noticed. Unfortunately, due to the nature of the R programming language, and the package being produced, FuzzyToolkitUoN also succumbed to one of these pitfalls - difficulty of use. Personally, I was frustrated with this, and that is one of the reasons for the birth of this project - remedying past mistakes.
\ \\
\ \\
As have been mentioned, the greater adoption of fuzzy logic would be extremely beneficial, as it adds a new level of reasoning that classical logic simply cannot. As such, another side goal of this project is to make a system that is as easy to use as possible, regardless of the skill of the user in both terms of knowledge of fuzzy logic, and of using computer software in general. This will mean a project that will not only be very easy to access and use, but also help novices to learn about what they are doing, as well as why they are doing it, to help them gain a greater understand of the field of fuzzy logic.
\ \\
\ \\
The project detailed in this dissertation will aim to implement a fuzzy logic software system, in a novel format (online), and to specifically avoid the common pitfalls observed of other similar systems. Being online, the system is already on the right path to solving the difficulty of access problem, as the users will be able to access the system from wherever they are, and on what ever platform (as it will not require any plugins, like Java, or Flash). This also means it is more accessible to the novice user, or the computer novice, as they need only navigate to a website to use the system; there is no complex download and installation process.
\ \\
\ \\
The project will be heavily influenced by the field of Human Computer Interaction, to ensure that the system is as user-friendly, and easy to pick up as possible. User interaction with a software system is extremely important, and the way systems are designed has a huge impact on how they are received by the user base. Simplicity is important in this design, because studies have shown that users lose more than 40\% of their time to frustration, and that in most of these cases, the user ends up angry at themselves, angry at the computer, or feeling a sense of helplessness \cite{lazar2006workplace}; which is obviously not ideal for a system that is attempting to help the user learn.