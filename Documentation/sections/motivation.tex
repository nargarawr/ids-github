\section{Motivation}
The motivation of this project is a simple one: to produce a fuzzy logic software system that is easy to access, and easy to use, to help promote the wider adoption of fuzzy logic. The problem with systems currently available is that they suffer from one of the two following pitfalls: difficulty of use and difficulty of access. This means that novices can find it very difficult to get into the field, the software available does not facilitate productive use, and even experts can be held back by the software they are using. Some specific issues include: locating systems to use, complex installation processes, cost to the user, unintuitive user interface, or a requirement of (a considerable amount of) prior knowledge. 
\ \\
\ \\
As part of my second year group project at the University of Nottingham, I worked on an R Package called FuzzyToolkitUoN. The goal of this system was to expand upon work completed by the IMA group\footnote{http://ima.ac.uk/}, to facilitate the use of fuzzy logic within the R programming language. Whilst working on this project, my group and I conducted a large amount of research into existing fuzzy logic software systems and it was during this research period that I began to notice the two keys flaws that I have mentioned before. Unfortunately, due to the nature of the R programming language, and the package we were producing, our project too fell into one of these pitfalls - difficulty of use. 


{\color{red} Motivation explaining the problem being solved}

{\color{red} why it is novel/not trivial }