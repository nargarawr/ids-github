\documentclass[a4paper,twoside,notitlepage,11pt]{article}
\usepackage{graphicx}
\usepackage{hyperref}
\usepackage{fancyeq}
\usepackage{amstext}
\usepackage{tabularx}
\usepackage{amssymb}
\usepackage{fancyhdr}
\usepackage{multirow}
\usepackage{tocloft}
\usepackage{lscape}
\usepackage{subcaption}
\usepackage{rotating}
\usepackage{titletoc}
\usepackage{setspace}
\usepackage[all]{xy}
\usepackage{nccmath}
\usepackage[a4paper,hmargin=2.5cm,vmargin=2.5cm]{geometry}
\usepackage[hypcap]{caption}
\usepackage{qtree}
\usepackage{cleveref}
\usepackage{mathpartir}
\usepackage{listings}
\usepackage{float}
\usepackage[compact]{titlesec}
\usepackage{layouts}
\usepackage{wrapfig}
\usepackage{listings}
\usepackage{color}
\usepackage{ulem}

\hypersetup{
    unicode=false,
    pdftoolbar=true,  
    pdfmenubar=true,     
    pdffitwindow=false,    
    pdfstartview={FitH},  
    pdftitle={Title},  
    pdfauthor={Craig Knott},
    pdfsubject={CS},
    pdfnewwindow=true,  
    colorlinks=true,  
    linkcolor=black,  
    citecolor=black, 
    filecolor=black,  
    urlcolor=black  
}

\newcommand{\paperTitleShort}{Web-based Fuzzy Logic Visualisation and Inferencing System}
\newcommand{\paperTitle}{Improving usability and accessibility of Fuzzy Logic software systems with a web-based approach}

\setlength{\parskip}{0.25em}

\renewcommand{\headrulewidth}{0.0pt}

\fancypagestyle{plain}{
	\fancyhf{}
	\fancyhead[LO]{Craig Knott}
	\fancyhead[CO]{\textit{\paperTitleShort}}
	\fancyhead[RO]{\thepage}
	\fancyhead[LE]{\thepage}
	\fancyhead[CE]{\textit{\paperTitleShort}}
	\fancyhead[RE]{Craig Knott}
}

\fancypagestyle{appendix}{
	\fancyhf{}
	\fancyhead[LO]{Craig Knott}
	\fancyhead[CO]{\textit{\paperTitleShort}}
	\fancyhead[RO]{Appendix \thepage}
	\fancyhead[LE]{Appendix \thepage}
	\fancyhead[CE]{\textit{\paperTitleShort}}
	\fancyhead[RE]{Craig Knott}
}

\fancypagestyle{biblio}{
	\fancyhf{}
	\fancyhead[LO]{cxk01u}
	\fancyhead[CO]{\textit{\paperTitleShort}}
	\fancyhead[RO]{\thepage}
	\fancyhead[LE]{\thepage}
	\fancyhead[CE]{\textit{\paperTitleShort}}
	\fancyhead[RE]{Craig Knott}
}

\begin{document}

%
%
% Title Page
%
%

\pagestyle{empty}
\newgeometry{left=99pt, right=99pt, top = 75pt, bottom=75pt}
\input{sections/titlepage}

%
%
% Abstract
%
%

\newpage
\begin{abstract}
{\color{red} Abstract giving a short overview of the work in your project
\\
\\
Why did you undertake the study? 
What were you examining, or testing or investigating. 
Return to your research question and ensure you have re-stated it concisely, coherently and clearly. A good opening is often, "The report examines . . . ".
\\
What was done and how did you do it? Be specific, don’t make generalised comments. 


What did you find out? State specific outcomes and, if appropriate, draw conclusions. ``The results found that 85\% of respondents used non-standardised assessments''

}
\end{abstract}

\begin{large}
Ensure to mention:
\begin{enumerate}
	\item What is fuzzy
	\item Further work with Type 2
	\item Extend with other systems (Joe's dissertation)
	\item Changes that Luke makes
	\item Friendly errors
	\item Things from presentation
	\item KeyPress Javascript library
	\item Help system is dedicated, but offers links to other, helpful, external resources
	\item only the evaluation step requires the internet (other than initial launch)
\end{enumerate}


\end{large}

%
%
% Table of Contents
%
%

\newpage
\newgeometry{left=3cm, right=3cm, top = 75pt, bottom=75pt}
\addtocontents{toc}{\protect\thispagestyle{empty}}
\tableofcontents

%
%
% Sections 
%
%

\newpage
\pagestyle{plain}
\setcounter{page}{1}

\section{Introduction}
{\color{red}setting out the aims and objectives of your project}

\section{Motivation}
The motivation of this project is a simple one: to produce a fuzzy logic software system that is easy to access, and easy to use, to help promote the wider adoption of fuzzy logic. The problem with systems currently available is that they suffer from one of the two following pitfalls: difficulty of use and difficulty of access. This means that novices can find it very difficult to get into the field, the software available does not facilitate productive use, and even experts can be held back by the software they are using. Some specific issues include: locating systems to use, complex installation processes, cost to the user, unintuitive user interface, or a requirement of (a considerable amount of) prior knowledge. 
\ \\
\ \\
As part of my second year group project at the University of Nottingham, I worked on an R Package called FuzzyToolkitUoN. The goal of this system was to expand upon work completed by the IMA group\footnote{http://ima.ac.uk/}, to facilitate the use of fuzzy logic within the R programming language. Whilst working on this project, my group and I conducted a large amount of research into existing fuzzy logic software systems and it was during this research period that I began to notice the two keys flaws that I have mentioned before. Unfortunately, due to the nature of the R programming language, and the package we were producing, our project too fell into one of these pitfalls - difficulty of use. 


{\color{red} Motivation explaining the problem being solved}

{\color{red} why it is novel/not trivial }

\section{Background Information \& Research}

{\color{red}Research section}
\subsection{What is Fuzzy Logic?}
\subsection{Existing Systems}
\label{sec:existing-systems}

{\color{red}\begin{enumerate}
\item Existing systems\\
{\color{red} Related work explaining what your project does that is new or is better than existing work in the same field}
	\begin{enumerate}
		\item FuzzyToolkitUoN
		\item MATLAB
		\item otheres... (see pres)
	\end{enumerate}
\end{enumerate}
}

\subsection{Platforms and Tools}
{\color{red}\begin{enumerate}
\item Languages used (R, Javascript)
\item Web technologies (tools, languages)
\item Shiny r to html
\item Bootstrap
\item jquery
\item Good user interfaces
\end{enumerate}}



\section{System Specification}
{\color{red} Description of the work explaining what your project is meant to achieve, how it is meant to function, perhaps even a functional specification}
	\subsection{Functional Requirements}
	\subsection{Non-Functional Requirements}

\section{Designs}
{\color{red} Design containing a comprehensive description of the design chosen, how it addresses the problem, and why it is designed the way it is}

\subsection{The Design Process}
{\color{red}  }


\subsection{UI Design}
{\color{red}  }


	\subsubsection{First Iteration}
	{\color{red}  }



	\subsubsection{Second Iteration}
	{\color{red}  }


	\subsubsection{Third Iteration}		
	{\color{red}  }


\subsection{Navigation/Control Flow Design}		
{\color{red}  }


\subsection{Internal Design}
{\color{red}  }


\section{Software Implementation}

\subsection{Key Implementation Decisions}
\label{sec:kid}
{\color{red} A list of the tools and platforms that were used in the project, including justification for their inclusion. (To cover: Languages used, Web technologies used, Shiny, Bootstrap, JQuery)}


\subsection{Detailed Description of the Implemented System}
{\color{red} A detailed description of the system, including how the individual sections were implemented, how they interact, strengths and weaknesses of individual components. Talk about the system as a whole as well, covering how the parts interacted and how well they did so.}


\subsection{Problems Encountered}
{\color{red} 
A description of any major problems encountered during the implementation of the project, causes of these, and their resolutions - if there were any. Talk about the impact of the problems on the project as a whole, and how this potentially affected it (like changes to designs, etc)
}

\section{Evaluation of the Project}
Testing and evaluation were vital parts of the project, to ensure that the system produced, both contained all the functionality it should have, it adhered to all the constraints place upon it and that it was actually usable. Three stages of testing were therefore conducted. The first of these was functionality testing, in which the functional requirements of the system would be evaluated, and the system would be testing to see whether or not it implements this functionality. The next stage was non-functional testing, in which the non-functional requirements would be evaluated in turn, to ensure the system abided by them. The final stage was user feedback testing, in which a group of participants would actually be using the system. In this stage, three systems would be evaluated: MATLAB, FuzzyToolkitUoN, and this project; in order to determine which provided a better user experience.

\subsection{Functional Testing}
In this section, each of the functional requirements laid out in section \ref{sec:funcs} have been evaluated in turn, to ensure the system meets them. Knowledge of the inner workings of the system is not actually necessary to understand these tests, as they simply check whether functionality is present, and are not concerned as to how the system actually implements it (this is known as black box testing \cite{beizer1995black}). A complete listing of all the tests conducted, and their results, can be found in appendix \ref{app-ctl}.

\subsection{Non-Functional Testing}

\subsection{User Feedback Testing}
\label{sec:uft}

% mention how someone obviously said the errors were friendly

% as part of jakon nielsons usability engineering life cycle 
\cite{nielsen1992usability}
% i did some user evaluation%%%%%%%%%

%	{\color{red}
%		Comparisons against other software, and checking against non-functional requirements. General explanation about the tests, and the participants, and what we are looking for (non functional, usability, accessibility). talk about using disseration to do fuz coursework. ``Considering I wrote both FTU and oFuzz, i found oFuzz much easier to use. Having the fuz coursework gave me a unique experience to test using both systems, as a user, instead of as the developer.''
		
%		{\color{blue}
%		An important part of evaluating the usability and accessibility of my system was to have real world users attempt to actual use not only my system, but similar systems, so that they could make comparative comments. In order to do this, I set up several sessions in which I would invite users of various skills levels, both in terms of computers, and fuzzy logic, to complete a list of tasks using all three software systems, after which I would ask them to give their feedback and opinions on all the systems. The three software systems that were evaluated were: my project, the MATLAB fuzzy toolbox, and FuzzyToolkitUoN, within the standard R environment.\ \\
%		\ \\
%		I split the participants into four main categories, based on their skill levels in terms of using computers, and knowledge of fuzzy logic. There were a total of 23 participants in these studies, and the distribution of skill levels is displayed in figure \ref{fig-skills}. The reason for this split was so that these distinctive groups could be evaluated individually, and their specific requirements could be observed. For instance, a participant skilled in computers, but not in fuzzy logic, would not struggle in navigating a system, but could potentially struggle understanding some of the fuzzy terminology.
		
%		\begin{figure}[ht!]
%		\begin{center}
%		\begin{tabular}{cc|cc}
%			& &\multicolumn{2}{c}{Computer Skill} \\
%			& & Low & High \\
%			\hline 
%		    \multirow{2}{2cm}{Fuzzy Logic Skill}  & Low & 7 & 5  \\
%		     &  High                                    & 3 & 8  \\
%		     \hline
%		     \\
%		     \multicolumn{3}{r}{\textbf{Total}} & 23\\
%		\end{tabular}
%		\end{center}
%		\vspace{-5mm}
%		\caption{Distribution of participant skill levels}
%		\label{fig-skills}
%		\vspace{-2mm}
%		\end{figure}
%		\noindent 
%		The task assigned to the participants was designed to use as much of the different systems as possible, but focused mainly on cross-compatible parts of the systems, so they could be easily compared. The test itself was to construct the fuzzy tipper example, using service and food as inputs, to produce a number for the tip to leave (the full set of instructions can be found in appendix \ref{app-userEval}). 
%		}
%	}
	\subsubsection{Evaluation of FuzzyToolkitUoN}
%		{\color{red}
%			Good things, bad things, statistics to back this up, talk about non funcs, and mention each type of user
%		}
	\subsubsection{Evaluation of MATLAB Fuzzy Toolbox} 	
%		{\color{red}
%			As above, but comparisons with above. specificaly mention that MATLAB has milliions of windows to open which make it very confusing!!!		
%		}
	\subsubsection{Evaluation of My Project}	
%		{\color{red}
%			As above, but comparison with above and above above
%		}
	\subsubsection{Summary}	
%		{\color{red}
%			Overall results and comparative statistics. 
%			{\color{blue}
%			The main two factors that were observed whilst the participants were completing the tasks were the speed at which they could do so, and the ease. Generally a faster completion meant either a high level of understanding, or an easier piece of software to use. The data collected strongly suggests that completion of the task using FuzzyToolkitUoN was the most difficult, which after speaking to the participants was the result of a poor user interface, and a very steep learning curve (especially for those of a novice computer skill level). The graph in figure \ref{fig:times} shows box plots of the completion times of each of the tasks, for each of the different groups. Each of these plots shows that FuzzyToolkitUoN was the most time consuming task to complete (taking on average {\color{red} 100 seconds}), and that the graphical systems were much easier to use (with MATLAB on average, taking {\color{red} 40 \%} less time, and my new system taking {\color{red} 10 \%} less time than that).
			
%			\begin{figure}[ht!]
%			\begin{center}
%			some graph yo
%			\end{center}
%			\vspace{-5mm}
%			\caption{Time taken to complete the tasks in the different software systems}
%			\label{fig:times}
%			\vspace{-2mm}
%			\end{figure}
			
%			Whilst the time taken to complete the tasks was a strong indicator of the success of the software system, it was also important to ask the participants which system they enjoyed using the most. The results for this were conclusive, which 100\% of participants (across all categories) claiming FuzzyToolkitUoN was the piece of software they enjoyed using the least. {\color{red} This is because...}. The piece of software that the participants enjoyed using the most was {\color{red} x \%} in favour of my produced software, over MATLAB (and the majority of those that said they preferred MATLAB said so as they were already very familiar with MATLAB). The results for most favoured,  software system can be seen in figure \ref{fig:mostleast}, organised by category of participant. {\color{red} The main reasons for an attraction to MATLAB were... and o-fuzz}
			
%			\begin{figure}[ht!]
%			\begin{center}
%			\includegraphics[width=0.8\textwidth]{images/graphsSmall.png}
%			\end{center}
%			\vspace{-5mm}
%			\caption{Favoured/Least Favoured software system, by participant category}
%			\label{fig:mostleast}
%			\vspace{-2mm}
%			\end{figure}
%			}
%		}		

\subsection{Successes and Limitations of the Project}
%{\color{red} 
%Explain what went well with the project, what didn't go so well, and what could be done better for next time (with concrete goals). issues with FTU as the backend (some options in params disabled because they don't work in FTU - not my fault!).

%}

\section{Further Work}
{\color{red}further... work?}

\subsection{Type-2 Fuzzy Logic}
\label{sec:type2}

{\color{red} brief explanation of what it is, why it is different to type1, and why it was not included}

\subsection{Backend Interoperability}
{\color{red} mostly in javascript, so only the inference engine needs to be ported}


\subsection{Customisations}

{\color{red} 
nothing major, but small things to make people feel more at home if they are using the system a lot}



\section{Summary and Further Work}
{\color{red}Summary and further work including a personal reflection on your experience of the project and a critical appraisal of how the project went}

\null
\vfill
\begin{flushright}
	\begin{tabular}{r|l}
		Words in text	& 1 \\
	\end{tabular}
	\ \\
	\ \\
	\begin{tabular}{cc}
		\multicolumn{2}{r}{Calculated with the TeXCount web service}\\
		\multicolumn{2}{r}{\url{http://app.uio.no/ifi/texcount/online.php}}
	\end{tabular}
\end{flushright}

%
%
% Bibliography
%
%

\newpage
\pagestyle{biblio}
\bibliographystyle{plain}

\bibliography{bib}


%
%
%  Appendix 
% 
%
\newpage
\setcounter{page}{1}
\pagestyle{appendix}
\appendix
\section{User Evaluation Test Instructions}
\label{app-userEval}
The purpose of this test is to construct the fuzzy logic tipper example, in three different software systems, so that the strengths and weaknesses of each of these systems can be identified. The tipper example is a simple fuzzy system that uses food quality (rated from 0 to 10) and service quality (rated from 0 to 10) as inputs, to determine how much of a tip should be left (between 0 and 30 percent). \ \\
\ \\
Please follow the instructions below, and do not hesitate to ask for help if you are stuck (this is not a test!)

\begin{enumerate}
\item test
\end{enumerate}


\section{Table of Results of User Evaluation}


%
%
% Qed
%
%

\end{document}








	