\documentclass[a4paper,twoside,notitlepage,11pt]{article}
\usepackage{etex}
\usepackage{graphicx}
\usepackage{hyperref}
\usepackage{fancyeq}
\usepackage{amstext}
\usepackage{tabularx}
\usepackage{amssymb}
\usepackage{fancyhdr}
\usepackage{tocloft}
\usepackage{lscape}
\usepackage{titletoc}
\usepackage{setspace}
\usepackage[all]{xy}
\usepackage{nccmath}
\usepackage[a4paper,hmargin=50pt,vmargin=75pt]{geometry}
\usepackage[hypcap]{caption}
\usepackage{cleveref}
\usepackage{mathpartir}
\usepackage{float}
\usepackage[compact]{titlesec}
\usepackage{layouts}
\usepackage{wrapfig}
\usepackage{listings}
\usepackage{color}
\usepackage{pgfgantt}

\hypersetup{
    unicode=false,
    pdftoolbar=true,  
    pdfmenubar=true,     
    pdffitwindow=false,    
    pdfstartview={FitH},  
    pdftitle={Title},  
    pdfauthor={Craig Knott},
    pdfsubject={CS},
    pdfnewwindow=true,  
    colorlinks=true,  
    linkcolor=black,  
    citecolor=black, 
    filecolor=black,  
    urlcolor=black  
}

\newcommand{\paperTitleShort}{Web-based Fuzzy Logic Visualisation and Inferencing System}
\newcommand{\paperTitle}{Project Proposal}

\setlength{\parskip}{0.25em}

\renewcommand{\headrulewidth}{0.0pt}

\fancypagestyle{plain}{
	\fancyhf{}
	\fancyhead[LO]{Craig Knott}
	\fancyhead[CO]{\textit{Web-Based Fuzzy Logic Visualisation and Inferencing System}}
	\fancyhead[RO]{\thepage}
	\fancyhead[LE]{\thepage}
	\fancyhead[CE]{\textit{Project Proposal}}
	\fancyhead[RE]{Craig Knott}
}

\begin{document}
%=====================================================================                                                       
%   _______ __  __        ____                 
%  /_  __(_) /_/ /__     / __ \____ _____ ____ 
%   / / / / __/ / _ \   / /_/ / __ `/ __ `/ _ \
%  / / / / /_/ /  __/  / ____/ /_/ / /_/ /  __/
% /_/ /_/\__/_/\___/  /_/    \__,_/\__, /\___/ 
%                                 /____/       
%=====================================================================                                                       
% http://patorjk.com/software/taag  Font: Slant

\pagestyle{empty}
\begin{center}
 {\LARGE \textbf{Project Proposal} \\ [0.2cm]}
 \textbf{G53IDS - 40 Credits}\\
   \textbf{Title: Web Based Fuzzy Logic Inferencing and Visualization System}\\
    \textbf{Craig Knott (cxk01u)} \\
	 \textbf{\today}
\end{center}

\section{Project Background}
As part of my second year group project, I was required to write a library to facilitate the usage of Fuzzy Logic within the R programming language. Fuzzy Logic is a method of modelling problems that is not restricted to crisp boolean values of true and false, which allows for work in areas of imprecision and uncertainty. Whilst working on this project and researching existing Fuzzy Logic software, it became quickly apparent that there was one key flaw shared by all Fuzzy Logic software. That is, it is not easily accessible, and it is not easily usable. By this I mean that of all the software that I looked at, including the one developed for my group project, there was not a single one that had both a quick installation time, and an easy to use interface. The main Fuzzy Logic software tools available at the moment are FuzzyToolkitUoN (the project I participated in), and MATLAB. The mains problems with FuzzyToolkitUoN are that it is entirely ran through the R interpretor, which requires knowledge of R and installing R packages, and requires the use of the command line, which can make using the system rather troublesome (FuzzyToolkitUoN does have a graphical counterpart, but this is written in Java and thus requires even more software to be downloaded just to begin using the system). MATLAB, on the other hand, has both a graphical interface and a command line interface, but these are both small parts of the overall MATLAB suite; they are not dedicated tools, and, of course, MATLAB is commercial and requires a purchase prior to use. \ \\
\ \\
This brings me onto my project; a \emph{web-based} fuzzy logic inferencing and visualization system. My main focus with this project is to create an easily accessible and easily usable software for working with Fuzzy Logic. This is why this project will be web-based; this means that users not be required to install anything in order to begin using the system, and as soon as they have navigated to the web-page, will be able to begin creating their systems. My project will aim to provide all the basic functionality that the FuzzyToolkitUoN package provides (creation and evaluation of Type-1 Fuzzy Systems), but, as stated, in a web format. In order to achieve this, I will be looking into services that will allow direct use of R packages in an HTML web-page, meaning my project will essentially be a user friend front-end for the R package I produced last year, written in HTML, CSS and JavaScript.

\section{Aims \& Objectives}
The main aim for the project is to produce a user friendly web application for working with Fuzzy Logic. This will be further broken down into more manageable subtasks, which have been listed below:
\begin{enumerate}
\item Produce an easily accessible system for working with fuzzy logic, with a focus on ease of use
\begin{enumerate}
\item Produce a user friendly membership function creation system
\item Produce a user friendly fuzzy variable creation system
\item Produce a user friendly fuzzy rule creation system
\item Add support for loading and saving of previous fuzzy systems
\item Allow for the evaluation of fuzzy systems and the display of results in a intuitive manner
\item Produce a non-intrusive help system that will guide the user as they are using the system - if they so desire
\item Research and investigate whether the system truly has improved ease of usability of fuzzy systems, in comparison to other software
\end{enumerate}
\end{enumerate}

\newpage
\pagestyle{plain}
\section{Work Packages \& Deliverables}
Below I have listed to various segments of work that will be completed, including what each one will entail, and the desired deliverable.\ \\
\ \\
\noindent
\textbf{Membership Function Creation System}\ \\
Membership functions form the main part of any fuzzy system. They are the building blocks of variables, which go onto be the building blocks of the system as a whole. The membership function creation system will allow the users of the system to specify various types of membership functions, and modify the values and bounds of them. This particular subsystem will be accompanied by a live-updating graphical representation of the function, so that the user can see their function changing in real time, to make sure it is exactly how they require.
\ \\
Deliverable: Fully functional system that allows for the specification of various membership functions, with a live updating graphical representation of these membership functions.\ \\
\ \\
\noindent
\textbf{Variable  Creation System}\ \\
Made up of membership functions, variables dictate what will be input and output by the system. This part of the system will be used to create these variable structures, specify their bounds, and allow users to add the membership functions required. They will then be organised in such a way that they are easy to view and manage, so the user can rectify any mistakes, or change the system in any way they wish. 
\ \\
Deliverable: Fully functional system that allows for the specification of fuzzy variables (both input and output), with the ability to add membership function to these.\ \\
\ \\
\noindent
\textbf{Rule Creation System}\ \\
Fuzzy rules are the bridge between the inputs and the outputs of the system, following a ``If this input has this value then that output will have this value'' pattern. As this is a relatively strict pattern, a simple graphical interface for constructing these rules will be produced, to make this process as simple and intuitive as possible,
\ \\
Deliverable: A simple and intuitive system for constructing fuzzy rules, using the inputs and outputs provided.\ \\
\ \\
\noindent
\textbf{File Manager}\ \\
A simple system for displaying the system as a whole in an easy to read text format, saving these files in a variety of formats, and loading in files that have been exported and loading them into the system. This section of the project will simply allow for users to save a project they are working on, and allow for the loading of previously worked on projects.
\ \\
Deliverable: A simple file saving and loading system (for FuzzyToolkitUoN formats, MATLAB formats, and JSON formats).\ \\
\ \\
\noindent
\textbf{System Evaluator}\ \\
Fuzzy evaluation is the process of taking the inputs, outputs and rules of a fuzzy system, and combining these together. This evaluated system can take values as input, and translate these, with the rules of the system, into an output; the results of which will vary heavily on the system that has been designed. This section of the project will be a display of this process (so that the user can visualize what is taking place), and of course, where the results will be displayed. Any graphical aids that can be provided will also be present, although once again, this is heavily dependant on the system that has been produced.
\ \\
Deliverable: An evaluation section of the application, showing the result of the queries that the user has passed to the system, along with any possible visual aids\ \\
\ \\

\newpage
\noindent
\textbf{Help System}\ \\
Fuzzy logic can sometimes be difficult to understand, and this is not helped by having no consistency across fuzzy system creation services. This means that confusion can arise with the fuzzy system itself, and confusion can arise whilst trying to specify the fuzzy system in the creation software. In order to combat this, my system will focus heavily on ease of use, and the offering of help wherever possible. This will make the system as beginner friendly as possible, in regards to both people that are unfamiliar with fuzzy logic, and those that are unfamiliar with using web services. The mantra of this system is that it is to be ``non-intrusive''. By ``non-intrusive'', I mean that the help will be distributed in such a way that beginners will be able to access it with ease, but advanced users will not ever need see it, unless they wish to. I plan to have small help indicators scattered across the system that, when pressed, will help explain concepts of fuzzy logic, or how to use a particular feature of the system. However, if the button is not pressed, it will simply not do anything. This means that beginners can ask for additional help, where as the advanced users can work in peace.
\ \\
Deliverable: Help available at any point in the system, but only if requested by the user.\ \\
\ \\
\noindent
\textbf{Research and Investigation into Ease of Use}\ \\
The final task of my project will be to actually evaluate its usefulness. I will conduct research that aims to answer a simple question: ``Is my system considerably easier to access and use than other systems available at the current time''. The purpose of this is to critically evaluate the system that I have produced, and to receive feedback on it from test subjects of all skill levels and background.
\ \\
Deliverable: Detailed results report.
\newpage
%\usepackage[a4paper,hmargin=50pt,vmargin=75pt]{geometry}
\newgeometry{left=40pt,right=50pt,top=70pt,bottom=40pt}
\begin{landscape}
\begin{center}
	\begin{ganttchart}[
	y unit chart = 0.7cm,
	y unit title = 0.8cm,
	x unit = 0.57cm,
	vgrid,
	hgrid
	]{1}{32}
		\gantttitle{Project Time Plan}{32}\\ 
		\gantttitlelist{"Oct","Nov"}{4}
		\gantttitlelist{"Dec"}{5}	
		\gantttitlelist{"Jan","Feb"}{4}
		\gantttitlelist{"Mar"}{5}			
		\gantttitlelist{"Apr"}{4}	
		\gantttitlelist{"May"}{2}\\
	    \gantttitlelist{7,14,21,28,
	                    4,11,18,25,
	                    2,9,16,23,30,
	                    6,13,20,27,
	                    3,10,17,24,
	                    3,10,17,24,31,
	                    7,14,21,28,
	                    5,12}{1}\\
			\ganttbar{Look up basics of web scripting}{1}{1}\\
			\ganttbar{Look for supplementary HTML services}{2}{2}\\		
			\ganttbar{Recap fundamentals of fuzzy logic}{3}{3}\\	
			\ganttbar{Recap principles of HCI}{4}{4}\\									
			\ganttbar{Membership function creator}{4}{6}\\		% mf creator
			\ganttbar{Input variable creator}{7}{9}\\		% input var
			\ganttbar{Output variable creator}{10}{11}\\	% output var
			\ganttbar[
				bar/.append style={fill=gray}
			]{Christmas break/January Exams}{12}{16}\\
			\ganttbar{Rule creator}{17}{18}\\	% rule creator
			\ganttbar{System evaluator}{19}{21}\\	% evaluation
			\ganttbar{Evaluator supplements}{22}{23}\\	% eval supp
			\ganttbar{File Saving/Loading}{19}{20}\\
			\ganttbar{Help system}{5}{21}\\		% help system

			\ganttbar{Research ease of use of other systems}{20}{22}\\	
			\ganttbar{Research ease of use of my system}{22}{24}\\	
			\ganttbar{Document findings of investigations}{25}{26}\\
			
			\ganttbar{Dissertation introductory sections}{5}{6}\\		% title, intro, abs, ack
			\ganttbar{Dissertation main body}{18}{29}\\		% main body
			\ganttbar{Dissertation conclusions}{30}{31}\\		% conclusions
			\ganttbar{Dissertations appendices/references}{18}{31}\\		% ref, apndx, figures
			\ganttmilestone{Hand-in week}{31}
			
			\ganttlink{elem8}{elem11}
			\ganttlink{elem4}{elem5}
			\ganttlink{elem4}{elem10}
			\ganttlink{elem4}{elem6}			
			\ganttlink{elem5}{elem8}			
			\ganttlink{elem6}{elem8}
			\ganttlink{elem8}{elem9}
			\ganttlink{elem9}{elem10}
			\ganttlink{elem10}{elem14}
			\ganttlink{elem13}{elem15}			
			\ganttlink{elem14}{elem15}
			\ganttlink{elem15}{elem18}			
			\ganttlink{elem17}{elem18}
			\ganttlink{elem18}{elem20}
	\end{ganttchart}
\end{center}
\end{landscape}
\end{document}